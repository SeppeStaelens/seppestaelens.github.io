\documentclass{article}

\usepackage{graphicx} % Required for inserting images
\usepackage{amsmath} % math
\usepackage{amssymb}
\usepackage{amsthm}

\title{Workshop}
\author{Seppe Staelens}
\date{\today}

\begin{document}

\maketitle

\section{Introduction}

Here I am introducing my project.

This is a new line.

\noindent This is a new line without indent.\\
This is a new line as well, also without indent.\\

This is a new paragraph\\
\\
This is a new paragraph without indentation.

\subsection{part of intro}

\subsubsection{very small detail}

\section{Main stuff}

\subsection{Mathematics}

To write maths, we need to enter \textit{math mode}. This is \textbf{very important}.

The basic way to enter math mode is by inserting maths between dollar signs. So I can now write $x = 4$. \\
Notice the difference without dollar signs, x = 4. If we have a longer equation, that we want to centralize, we can use double dollar signs like this:
$$\int_0^{+\infty} e^{-x} \text{d} x = 1$$

Suppose I have a very important equation, that I want to refer to later.
Also
\begin{equation}
    a^p \equiv a \mod p
\end{equation}
\begin{equation}\label{eq: Pythagoras}
    a^2 + b^2 = c^2
\end{equation}
Look at equation (\ref{eq: Pythagoras}).
\newpage
If I have a calculation to do, and I want multiple lines of math mode, I can do
\begin{align*}
    x & = (a+b)^2 \\
    \Leftrightarrow x & = a^2 + 2ab + b^2
\end{align*}

\begin{proof}
    I will prove the Riemann hypothesis.

    Done
    $\mathbb{R}$ $\# \mathbb{R} = 2^{\aleph_0}$
    $A = \{1,2,3\}$
\end{proof}


\end{document}
